\chapter{Opérations sur les tables de données\label{c:tables}}

\section{Travail sur les lignes et colonnes}

Dans une grande variété de situations, il peut être avantageux de répéter une opération sur toutes les lignes, ou toutes les colonnes.
R propose une fonction pour automatiser ce traitement, \emph{via} la fonction \texttt{apply}. 

\begin{knitrout}
\definecolor{shadecolor}{rgb}{1, 1, 1}\color{fgcolor}\begin{kframe}
\begin{flushleft}
\ttfamily\noindent
\hlsymbol{dat}{\ }\hlassignement{\usebox{\hlnormalsizeboxlessthan}-}{\ }\hlfunctioncall{matrix}\hlkeyword{(}\hlfunctioncall{rnorm}\hlkeyword{(}\hlnumber{100}\hlkeyword{)}\hlkeyword{,}{\ }\hlargument{nrow}{\ }\hlargument{=}{\ }\hlnumber{10}\hlkeyword{)}\hspace*{\fill}\\
\hlstd{}\hlfunctioncall{apply}\hlkeyword{(}\hlsymbol{dat}\hlkeyword{,}{\ }\hlnumber{1}\hlkeyword{,}{\ }\hlsymbol{mean}\hlkeyword{)}\mbox{}
\normalfont
\end{flushleft}
\begin{verbatim}
##  [1]  0.3636 -0.1601 -0.1409  0.2279  0.7192  0.2508  0.3597  0.4507
##  [9] -0.2436  0.4394
\end{verbatim}
\begin{flushleft}
\ttfamily\noindent
\hlfunctioncall{apply}\hlkeyword{(}\hlsymbol{dat}\hlkeyword{,}{\ }\hlnumber{2}\hlkeyword{,}{\ }\hlsymbol{var}\hlkeyword{)}\mbox{}
\normalfont
\end{flushleft}
\begin{verbatim}
##  [1] 0.6659 1.6678 0.7538 1.0510 0.4947 0.6434 1.2381 1.8382 1.2181 2.1166
\end{verbatim}
\end{kframe}
\end{knitrout}


\section{Traitement des données}

\section{Division et traitement par niveau}

En utilisant différentes fonctions, on peut traiter facilement un jeu de données par «niveaux» d'un facteur (p.ex. traitement expérimental).
En rechargeant les données \emph{Lamellodiscus}, on peut par exemple chercher à connaître la moyenne et la variance de la taille de chaque pièce sclérifiée.

\begin{knitrout}
\definecolor{shadecolor}{rgb}{1, 1, 1}\color{fgcolor}\begin{kframe}
\begin{flushleft}
\ttfamily\noindent
\hlsymbol{morpho}{\ }\hlassignement{\usebox{\hlnormalsizeboxlessthan}-}{\ }\hlfunctioncall{read.table}\hlkeyword{(}\hlstring{"{}data/lamellodiscus.txt"{}}\hlkeyword{,}{\ }\hlargument{h}{\ }\hlargument{=}{\ }\hlnumber{TRUE}\hlkeyword{,}{\ }\hlargument{sep}{\ }\hlargument{=}{\ }\hlstring{"{}\usebox{\hlnormalsizeboxbackslash}t"{}}\hlkeyword{)}\mbox{}
\normalfont
\end{flushleft}
\end{kframe}
\end{knitrout}


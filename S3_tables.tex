\chapter{Opérations sur les tables de données\label{c:tables}}

\section{Travail sur les lignes et colonnes}

Dans une grande variété de situations, il peut être avantageux de répéter une opération sur toutes les lignes, ou toutes les colonnes.
R propose une fonction pour automatiser ce traitement, \emph{via} la fonction \texttt{apply}. 

\begin{knitrout}
\definecolor{shadecolor}{rgb}{1, 1, 1}\color{fgcolor}\begin{kframe}
\begin{flushleft}
\ttfamily\noindent
\hlsymbol{dat}{\ }\hlassignement{\usebox{\hlnormalsizeboxlessthan}-}{\ }\hlfunctioncall{matrix}\hlkeyword{(}\hlfunctioncall{rnorm}\hlkeyword{(}\hlnumber{100}\hlkeyword{)}\hlkeyword{,}{\ }\hlargument{nrow}{\ }\hlargument{=}{\ }\hlnumber{10}\hlkeyword{)}\hspace*{\fill}\\
\hlstd{}\hlfunctioncall{apply}\hlkeyword{(}\hlsymbol{dat}\hlkeyword{,}{\ }\hlnumber{1}\hlkeyword{,}{\ }\hlsymbol{mean}\hlkeyword{)}\mbox{}
\normalfont
\end{flushleft}
\begin{verbatim}
##  [1] -0.026960 -0.015687  0.008188  0.089791 -0.421734  0.118884 -0.101086
##  [8] -0.169665 -0.052550  0.385399
\end{verbatim}
\begin{flushleft}
\ttfamily\noindent
\hlfunctioncall{apply}\hlkeyword{(}\hlsymbol{dat}\hlkeyword{,}{\ }\hlnumber{2}\hlkeyword{,}{\ }\hlsymbol{var}\hlkeyword{)}\mbox{}
\normalfont
\end{flushleft}
\begin{verbatim}
##  [1] 1.2223 1.4511 0.8813 0.7104 1.3745 0.7182 0.5745 0.7073 0.5857 1.3680
\end{verbatim}
\end{kframe}
\end{knitrout}


\section{Traitement des données}

\section{Division et traitement par niveau}
